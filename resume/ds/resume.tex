% LaTeX resume using res.cls
\documentclass[margin]{res}
\usepackage{amsmath}
\usepackage{amsfonts}
\usepackage{hyperref}
\usepackage{helvet}
\usepackage{xcolor}
\usepackage{enumitem}

\usepackage{xcolor}
\hypersetup{
    colorlinks,
    linkcolor={red!50!black},
    citecolor={blue!50!black},
    urlcolor={blue!80!black}
}

% \usepackage{helvetica} % uses helvetica postscript font (download helvetica.sty)
%\usepackage{newcent}   % uses new century schoolbook postscript font 
\setlength{\textwidth}{5.1in} % set width of text portion

\begin{document}

% Center the name over the entire width of resume:
 \moveleft.5\hoffset\centerline{\Huge\bf Julian M. Lehrer}
% Draw a horizontal line the whole width of resume:
 \moveleft\hoffset\vbox{\hrule width\resumewidth height 1pt}\smallskip
% address begins here
% Again, the address lines must be centered over entire width of resume:
% \moveleft.5\hoffset\centerline{628 Crown Road, Santa Cruz, CA 95064}
\moveleft.5\hoffset\centerline{}
\moveleft.5\hoffset\centerline{707-490-9354 $\mid$ \textcolor{blue}{\href{https://julianlehrer.me}{julianlehrer.me}} $\mid$ jmlehrer@ucsc.edu}

\begin{resume}
  
\section{EDUCATION}
	\textbf{University of California, Santa Cruz} \hfill {\sl Fall 2018 -  Spring 2021 (expected)} \\
	B.A. Computational Mathematics, Minor in Computer Science 

    \renewcommand\labelitemi{{\boldmath$\cdot$}}

\setlist{topsep=3pt}

\section{EXPERIENCE} 
    % \textbf{Software Engineering Intern} $\mid$ \textit{Valassis Digital |
    % Austin, TX} \hfill {\sl Summer 2021, hopefully}
    % \begin{itemize}
    %     \item Rewrote legacy data pipeline into Spark
    % \end{itemize} \vspace*{-10pt}
    \textbf{Data Science Intern} $\mid$ \textit{Blackthorn Therapeutics | San Francisco, CA} \hfill {\sl Summer 2020}
    \begin{itemize}
        \item Used statistical modeling to research the effects of isolation on depression and anxiety
        \item Wrote interpretable models in Python (scikit-learn) to be used in future clinical analysis
        \item Generated a research report and presentation for the company 
    \end{itemize} \vspace*{-10pt}

   \textbf{Data Science Intern} $\mid$ \textit{Startup Genome | San Francisco, CA} \hfill {\sl Spring 2020}
    \begin{itemize}
        % \item Built analytics pipeline for understanding how COVID-19 affects global startup ecosystems
        \item Created deep learning model with Python (Pandas, Tensorflow, NLTK) to classify startup sectors from funding data 
        % \item Collaborated on data engineering pipeline to prepare data for clients
        \item Wrote data engineering pipeline to generate and visualize funding metrics for clients
    \end{itemize} \vspace*{-10pt}
    % \textbf{Vice President} $\mid$ \textit{Data Science @ SC | Santa Cruz, CA} \hfill {\sl Winter 2020 - Current}
    % \begin{itemize}
    %     \item Organized outreach events, presented on Machine Learning techniques
    %     \item Created the UCSC Statistics Reading group
    % \end{itemize} \vspace*{-10pt}
\section{PROJECTS}
    \textbf{Project Portfolio} $\mid$ \textcolor{blue}{\href{https://github.com/jlehrer1/Projects}{https://github.com/jlehrer1/Projects}} \vspace {2mm} \\
    \textbf{Transparency Project (1st Place CruzHacks 2020)}
    \begin{itemize}
        \item A fully interactive website that brings clarity to the political process through interactive data visualizations. Build with Plot.ly and Dash, and hosted live on GCloud.
    \end{itemize}\vspace*{-8pt}
    \textbf{SQLtoPandas}
    \begin{itemize}
        \item Python package to use SQL querying on Pandas DataFrames without creating a SQL database
        \item Built with Python (pandas, numpy, sqlite3) published on PyPi
        % \item Created a classification system from the weather.gov AQI to determine the level of caution that should be taken outdoors
    \end{itemize}\vspace*{-8pt}

    \textbf{InstantEDA}
    \begin{itemize}
        \item Python package to instantly generate common exploratory data plots without cleaning your DataFrame
        \item Built with Python (pandas, numpy, plotly), published on PyPi
        % \item Created a classification system from the weather.gov AQI to determine the level of caution that should be taken outdoors
    \end{itemize}\vspace*{-8pt}
    \textbf{DrivenData: DengueAI}
    \begin{itemize}
        \item Used a combination of engineered lagged features and fourier models to achieve a top 11.8\% score globally (so far) on the DrivenData Dengue fever prediction contest
        \item Built with Pandas, Scikit-learn and Tensorflow
        % \item Created data visualizations using Plotly
    \end{itemize}\vspace*{-8pt}
\section{SKILLS}
    \textbf{Programming}: Python (scikit-learn, Pandas, Numpy, Tensorflow, Plotly), Swift, SQL, Java, C, C++, Matlab\\
    \textbf{Theory}: Statistical models, machine learning, deep learning, numerical optimization, numerical methods \\
    \textbf{Software}: AWS Elastic Beanstalk, AWS Lambda, Git, Bash 
\end{resume}
\end{document}
